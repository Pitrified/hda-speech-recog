% !TEX root = report.tex

\section{Introduction}
\label{sec:introduction}

% A fascinating introduction
% Never forget \cite{2018arXiv180808929C}

Keyword Spotting or Voice Command Recognition are key components that allow
speech based user interaction.
%
Many smart devices use a specific wake phrase to initiate the active
interaction with the user (e.g. ``Ok Google'', ``Alexa''), while other devices
are controlled by specific voice commands (e.g. ``play'', ``stop'', ``lights
off'').
%
Other automated services might respond to the user's voice, for example to
collect data during an automated telephone survey.
%
A pre-defined set of keywords must be identified within a continuos stream of
audio. Several properties are desired in the model that analyzes the audio: low
energy consumption to account for the ``always-on'' performance, that stands
alongside with a small memory footprint to allow the use of the model on
embedded devices, and with a low latency in the reaction, to provide a fluid
user experience.
%
These properties allow for the model to be run locally, and to only send audio
to a cloud-based system, with far higher computational capability, only after
waking up the device, to address the more complex interaction that follows.

% TODO: paper structure

% Control a device with voice command (play, stop, lights off)
% wake words
% initiate interaction
% continuos stream of audio
% predefined set of words
% auto no hands
% automated surveys
% low energy consumption
% low memory
% always on - cloud - latency - real time response
% low resources devices or micro-controllers

