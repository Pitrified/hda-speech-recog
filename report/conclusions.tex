% !TEX root = report.tex

\section{Concluding Remarks}
\label{sec:conclusions}

% A gripping conclusion
The task of identifing a command in a single audio sample has been successfully
solved, with varying degree of accuracy, by every architecture tested.

%
It was proved that the attention mechanism is not very effective when dealing
with spectrogram images.
%
A possible reason is that the relevant portion is already very clear, while in
the background regions that do not contain sound the spectrogram very uniform
and with very low values.
%
In these regions, the convolutional filters already have very low output
values, so using the attention mechanism to further emphasize the relevant
portions does not help, but can instead hinder the learning by mistakenly
dimming useful regions of the spectrogram.

% further work
Further works might focus on different types of attention, and the performance
of SimpleNet and AreaNet should be evaluated on standard image classification
datasets, such as ImageNet.
%
A more complex dataset might provide an insight to whether the proposed
attention mechanism is capable of learning to spot useful features when faced
with more evenly distributed input images, where the interesting part is not
immediately clear.
% can I spot the beak of the heron or do I just learn that the center is useful?

%
Having proved the effectiveness of SimpleNets, a transfer learning approach
that builds the AreaNet by using an SimpleNet already trained on the same task
as the final block (that is not modified in the first stage of learning) might
help, as the model only has to learn to spot the relevant portion of the data
in order to maximize the SimpleNet performance.
% And can that extractor be reused on a different task.

Yet another approach might be training the attention weight extractor by using
known object bounding boxes, making the model predict the boxes instead of the
image class, then using the predictions to weigh the input.

% TODO: after all learning attention weights is an *unsupervised* task
% as we do not know which regions are useful
% using object bounding boxes might be interesting first train the general
% feature extractor on loss measured against the boxes
% 0000000000
% 0001111000
% 0001111000
% 0000000000
% then use that to weigh the images

% TODO dighe che vuoi vedere se si rivela utile

Regarding the methodology used, a definite improvement would be using existing
frameworks for hyper-parameter tuning, such as Hypertune, readily available
with Keras Tuner library. A very interesting approach is Hyperband, that trains
every model built from the hyper-parameter grid for a few epochs, then prunes
the set of models, only keeping the best performing ones.

% TODO also much better preprocessing pipeline in ImageNetGenerator

% The methodical hyper-parameter analysis done while writing the report was very
% informative, and while the plots confirmed what I believed to be the best
% combinations, they gave me a couple of ideas regarding some explorations that I
% had missed.
