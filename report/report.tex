\documentclass[10pt, conference, letterpaper]{IEEEtran}

\usepackage{algorithm}
\usepackage{algorithmicx}
\usepackage{algpseudocode}
\usepackage{amsfonts}
\usepackage{amsmath}
\usepackage{amssymb}
\usepackage[ansinew]{inputenc}
\usepackage{xcolor}
\usepackage{mathtools}
\usepackage{graphicx}
\usepackage{caption}
\usepackage{subcaption}
\usepackage{import}
\usepackage{multirow}
\usepackage{cite}
\usepackage[export]{adjustbox}
\usepackage{breqn}
\usepackage{mathrsfs}
\usepackage{acronym}
%\usepackage[keeplastbox]{flushend}
\usepackage{setspace}
\usepackage{bm}
\usepackage{stackengine}

\usepackage{tikz}
\usetikzlibrary{calc}

\usepackage{listings}

\lstset{%
    backgroundcolor=\color[gray]{.85},
    basicstyle=\small\ttfamily,
    breaklines = true,
    keywordstyle=\color{red!75},
    columns=fullflexible,
}%

\lstdefinelanguage{BibTeX}
{keywords={%
        @article,@book,@collectedbook,@conference,@electronic,@ieeetranbstctl,%
        @inbook,@incollectedbook,@incollection,@injournal,@inproceedings,%
        @manual,@mastersthesis,@misc,@patent,@periodical,@phdthesis,@preamble,%
        @proceedings,@standard,@string,@techreport,@unpublished%
    },
    comment=[l][\itshape]{@comment},
    sensitive=false,
}

\usepackage{listings}

% listings settings from classicthesis package by
% Andr\'{e} Miede
\lstset{language=[LaTeX]Tex,%C++,
    keywordstyle=\color{RoyalBlue},%\bfseries,
    basicstyle=\small\ttfamily,
    %identifierstyle=\color{NavyBlue},
    commentstyle=\color{Green}\ttfamily,
    stringstyle=\rmfamily,
    numbers=none,%left,%
    numberstyle=\scriptsize,%\tiny
    stepnumber=5,
    numbersep=8pt,
    showstringspaces=false,
    breaklines=true,
    frameround=ftff,
    frame=single
    %frame=L
}

\renewcommand{\thetable}{\arabic{table}}
\renewcommand{\thesubtable}{\alph{subtable}}

\DeclareMathOperator*{\argmin}{arg\,min}
\DeclareMathOperator*{\argmax}{arg\,max}

\def\delequal{\mathrel{\ensurestackMath{\stackon[1pt]{=}{\scriptscriptstyle\Delta}}}}

\graphicspath{{./figures/}}
\setlength{\belowcaptionskip}{0mm}
\setlength{\textfloatsep}{8pt}

\newcommand{\eq}[1]{Eq.~\eqref{#1}}
\newcommand{\fig}[1]{Fig.~\ref{#1}}
\newcommand{\tab}[1]{Tab.~\ref{#1}}
\newcommand{\secref}[1]{Section~\ref{#1}}

\newcommand\MR[1]{\textcolor{blue}{#1}}
\newcommand\red[1]{\textcolor{red}{#1}}
\newcommand{\mytexttilde}{{\raise.17ex\hbox{$\scriptstyle\mathtt{\sim}$}}}

%\renewcommand{\baselinestretch}{0.98}
% \renewcommand{\bottomfraction}{0.8}
% \setlength{\abovecaptionskip}{0pt}
\setlength{\columnsep}{0.2in}

% \IEEEoverridecommandlockouts\IEEEpubid{\makebox[\columnwidth]{PUT COPYRIGHT NOTICE HERE \hfill} \hspace{\columnsep}\makebox[\columnwidth]{ }}

\title{Speech Command Recognition}

\author{pmn}

\IEEEoverridecommandlockouts

\newcounter{remark}[section]
\newenvironment{remark}[1][]{\refstepcounter{remark}\par\medskip
\textbf{Remark~\thesection.\theremark. #1} \rmfamily}{\medskip}

\begin{document}

\maketitle

\begin{abstract}
    This report covers the work done to solve the speech command recognition task,
    following three approaches:
    Convolutional Neural Networks,
    transfer learning using a pre-trained Xception architecture,
    and a recurrent attention model.

    A new method to compute the attention weight is proposed, using a Convolutional Neural Network to evaluate the scores, both directly from the spectrogram and from the LSTM features.

    Experiments to identify the best input preprocess and hyper-parameters are performed for the three architectures.

    Results are compared with the results obtained in the paper that proposed the attention model.
    (IN REALTA LORO SU TASK LEGGERMENTE DIVERSA)
\end{abstract}

\IEEEkeywords
Speech Command Recognition
\endIEEEkeywords

% !TEX root = report.tex

\section{Introduction}
\label{sec:introduction}

% A fascinating introduction
% Never forget \cite{2018arXiv180808929C}

Keyword Spotting is a key components that allows speech based user interaction.
%
Many smart devices use a specific wake phrase to initiate the active
interaction with the user (e.g. ``Ok Google'', ``Alexa''), while other devices
are controlled by specific voice commands (e.g. ``play'', ``stop'', ``lights
off'').
%
Other automated services might respond to the user's voice, for example to
collect data during an automated telephone survey.
%
A pre-defined set of keywords must be identified within a continuos stream of
audio. Several properties are desired in the model that analyzes the audio: low
energy consumption to account for the ``always-on'' requirement, that goes hand
in hand with a small memory footprint to allow the use of the model on embedded
or mobile devices, and with a low latency in the reaction, to provide a fluid
user experience.
%
These properties allow for the model to be run locally, and to only send audio
to a cloud-based system, with far higher computational capability, only after
waking up the device, to address the more complex interaction that follows.

% TODO: paper structure

% Control a device with voice command (play, stop, lights off)
% wake words
% initiate interaction
% continuos stream of audio
% predefined set of words
% auto no hands
% automated surveys
% low energy consumption
% low memory
% always on - cloud - latency - real time response
% low resources devices or micro-controllers



% !TEX root = report.tex

\section{Related Work}
\label{sec:related_work}

% MFCC + HMM
The first classification models used Mel Frequency Cepstral Coefficients to
compactly extract features from the audio signal and a Hidden Markov model to
compute the prediction, leveraging the HMM ability of modelling sequences
\cite{mfcchmm103088}.
% ConvNets
Deep Neural Networks, in more recent years, provided efficient solutions
\cite{chensmallDNN} and Convolutional Neural Networks have been successfully
applied to exploit the spatial correlations in the spectrograms while keeping a
small memory footprint for the model \cite{sainathconvolutional}.
% RNN
Long short-term memory cells are used to better deal with long time
dependencies within the signal, allowing some delay in the decision while
keeping an internal state that acts as memory \cite{fernandezRNNKWS}.

% Attention mechanism
The attention mechanism allows the model to selectively focus on some portions
of the input data that is judged to be more relevant.
% Attention wide variety of tasks 
This mechanism has been applied successfully to a variety of tasks, such as
machine translation \cite{luong2015effective}, \cite{bahdanau2016neural}, and
image caption generation \cite{xu2016show}.
% Andrade
The attention mechanism was adapted to the speech commands recognition task
\cite{2018arXiv180808929C}, where it was used in conjunction to LSTMs focusing
on single word recognition.
% Federated learning for privacy concerns
A federated learning approach can be used to train a central model on the local
data of many users \cite{leroy2019federated}, addressing concerns regarding the
privacy of the training data.
% Microcontrollers
The power and memory requirements, are carefully investigated in
\cite{zhang2018hello}, where a resource constrained neural network architecture
exploration is performed, to maximize the performances within a given set of
constraints. On top of that, Neural Network Quantization is used to shrink the
otained models even more, using the quantization technique presented in
\cite{10.1145/2847263.2847265}.
% Tsetlin Machines
A completely different learning framework is used in \cite{granmo2021tsetlin},
where a learning algorithm called the Tsetlin Machine is applied.
The core of this machine is the Tsetlin Automaton,
that uses a penalty/reward signal to train the automaton, whose decisions are
combined with a clause module to create logic propositions that describe the 
input features.
% A learning automata powered machine is used to 
Application-specific integrated circuit (ASIC) exist for this architecture,
allowing for extremely energy-efficient inference.
% Tsetlin machine (TM) is evaluated in the KWS-AI design in place of the
% traditional perceptron based NNs.
% The machine operates through deriving propositional logic that describes the
% input features.
% It has shown great potential over NN based models in delivering energy frugal AI application while maintaining faster convergence and high learning efficacy





% !TEX root = report.tex

\section{Processing Pipeline}
\label{sec:processing_architecture}

High level descripion of the processing flow:

Throughout the report: extract Spectrograms, classify image.

\section{Signals and Features}
\label{sec:model}

\subsection{Google Dataset}

% Measurement setup, data preprocess (describe the google dataset)

The Speech Commands datasets, described in \cite{warden2018speech}, provides
thousands of one second recording of $35$ differen words. The recordings are
mainly of volounteers, who used their own device to record the words in a
closed room wherever they happened to be (not in a studio setting).
Ideally each volounteer only recorded the $135$ requested utterances once, so the
dataset provides good variability of voices.
The utterances have a duration of one second.
To more precisely align the recorded clips, the audio was acquired for $1.5$
seconds and the $1$ second clip that contained the highest overrall volume was
extracted.
Several background noise recording are included as well.
The full dataset includes $105829$ utterances of $35$ words, saved in
\textit{.wav} format at $16$ KHz rate.
The dataset is released under the Creative Commons BY $4.0$ License \cite{ccby4}.

The dataset ships with a function to split the data in train, validation and
test folds, as well as two example lists of validation data ($9981$ utterances)
and test data ($11005$ utterances).
Throughout the experiments, those lists were used.
% both for the practicality of having them ready, but also to keep with the
% spirit of the dataset as a tool to enable meaningf

\subsection{Mel spectrogram and Mel-frequency Cepstral Coefficients}

The audio data in the dataset is available as a vector of amplitudes over time,
sampled at $16$ KHz. In this representation, the classification task is quite
hard.
The Fourier transform allows to convert a signal from the time domain into the
frequency domain. The result is called spectrum of the signal. This transform
is efficiently computed using the Fast Fourier Transform algorithm.

The short-time Fourier transform accounts for variations of the content of an
audio signal over time. Instead of computing the FFT of the entire signal, the
FFT is computed on overlapping windowed segments of the signal: each window has
length \texttt{n_fft} and the next window is extracted after
\texttt{hop_length} samples.
The results of the FFT in each window are stacked to obtain the spectrogram.

The human ear does not perceive frequencies on a linear scale: the difference
between $200$ and $400$ Hz is very marked, whereas two notes at $8000$ and
$8200$ Hz are almost indistinguishable. The mel scale, proposed by Stevens,
Volkmann, and Newmann \cite{melscale1937}, introduces a unit of pitch built in
such a way that equal distances on the scale sound equally distant to the human
listener.
The mel spectrogram is a spectrogram where the frequencies are converted to the
mel scale. To do so, a Mel spaced filterbank is generated (a 10 filters version
is shown in \fig{fig:mel10_filterbank}) and the FFT results are multiplied with
a dot product with each filter, obtaining \texttt{n_mel} values for each
timestep.

\begin{figure}[t!]
    \centering
    \includegraphics[width=0.8\linewidth]{mel10_filterbank.pdf}
    \caption{An example of a 10 filters Mel filterbank}
    \label{fig:mel10_filterbank}
\end{figure}

The resulting coefficients are highly correlated: the Discrete Cosine Transform
can be applied to decorrelate the filter bank coefficients and obtain a
compressed representation.
The results are the Mel-frequency Cepstral Coefficients.

An example waveform for the word ``happy'' and the relative spectrograms are
shown in \fig{fig:happy_specs}.

\begin{figure}[t!]
    \centering
    \includegraphics[width=0.8\linewidth]{happy_specs.pdf}
    \caption{Waveform and spectrograms for the word happy. Note that the y axis of the spectrograms are labeled as Hz, but this is only to read more easily the plot and understand to which frequencies the important bins correspond to.}%
    \label{fig:happy_specs}
\end{figure}

TODO: con che libreria sono implementati

TODO: che valori usati per generare gli spettrogrammi

TODO: che valori usati per generare l'augmentation

\subsection{Data augmentation}

Data augmentation is a technique to increase the amount of data available by
applying random, but meaningful, transformations to the data. This leads to a
noisier dataset, that should make the trained model more robust and less prone
to overfitting. The data was augmented both by modifying the waveform and the
spectrograms.

\subsubsection{Time shift}

The waveform is shifted by a random amount of samples, controlled by the
parameter \texttt{max_time_shift}.

\subsubsection{Time stretch}

The waveform is stretched, making the sound slower or faster, controlled by the
parameter \texttt{stretch_rate}.

\subsubsection{Spectrogram warp}

The spectrogram is warped using the \texttt{sparse_image_warp} function
available as a tensorflow addon.
A sequence of source points is randomly selected within the image, and the
points are shifted by a random amount along both time and frequency axis. The
warp is controlled by the parameters \texttt{num_landmarks},
\texttt{max_warp_time} and \texttt{max_warp_freq}.
An example is shown in \fig{fig:warp_grid}.

\begin{figure}[t!]
    \centering
    \includegraphics[width=0.8\linewidth]{warp_grid.pdf}
    \caption{Example of a \texttt{sparse_image_warp} with \texttt{num_landmarks} $=4$, \texttt{max_warp_time} $=2$ and \texttt{max_warp_freq} $=2$.}%
    \label{fig:warp_grid}
\end{figure}

\section{Learning Framework}
\label{sec:learning_framework}

How did it learn anything at all?

\subsection{Learning rate schedule}

\subsection{Hyper-parameter tuning}

\section{Convolutional Architecture}
\label{sec:convolutional_arch}

\section{Transfer Learning approach}
\label{sec:transfer_learning}

\section{Attention Model}
\label{sec:attention_model}

\subsection{Attention architecture}

\subsection{Query style}


% !TEX root = report.tex

\section{Results}
\label{sec:results}

Dazzling numerical results, plots to describe Fscore as function of the learning parameters.
Progressive and logical manner, starting with simple things and adding details.
Address one concept at a time.

Hyper-parameters, show selected results for several values of these.
Tables are a good approach to concisely visualize the performance as hyper-parameters change.
How architectural choices affect the overall performance.

Intro su come leggere i grafici, gruppi di gruppi di colonne.


% !TEX root = report.tex

\section{Concluding Remarks}
\label{sec:conclusions}

A gripping conclusion

Using Hypertune di Keras

Normalizing spectrograms might be useful


\bibliography{biblio}
\bibliographystyle{ieeetr}

\section{Appendix}
\label{sec:appendix}

% https://tex.stackexchange.com/a/210378
\renewcommand{\thefigure}{A.\arabic{figure}}
\setcounter{figure}{0}

\renewcommand{\thetable}{A.\arabic{table}}
\setcounter{table}{0}

Several large tables and figures are here for ease of reading.

\begin{figure}[t!]
    \centering
    \includegraphics[width=0.9\linewidth]{VAN_opa1_lr03_bs32_en14_dsaug14_wyn_cm.pdf}
    \caption{VAN opa1 lr03 bs32 en14 dsaug14 wyn cm}%
    \label{fig:VAN_opa1_lr03_bs32_en14_dsaug14_wyn_cm}
\end{figure}

\begin{figure}[t!]
    \centering
    \includegraphics[width=0.9\linewidth]{VAN_opa1_lr03_bs32_en15_dsaug14_wLTBnum_cm.pdf}
    \caption{VAN opa1 lr03 bs32 en15 dsaug14 wLTBnum cm}%
    \label{fig:VAN_opa1_lr03_bs32_en15_dsaug14_wLTBnum_cm}
\end{figure}

\begin{figure*}[t!]
    \centering
    \includegraphics[width=0.9\linewidth]{VAN_opa1_lr03_bs32_en15_dsmel04_wLTBall_noval_cm.pdf}
    \caption{VAN opa1 lr03 bs32 en15 dsmel04 wLTBall noval cm}%
    \label{fig:VAN_opa1_lr03_bs32_en15_dsmel04_wLTBall_noval_cm}
\end{figure*}

\begin{figure*}[t!]
    \centering
    \includegraphics[width=0.9\linewidth]{SI2_opa1_lr04_bs32_en15_dsmel04_wLTBall_noval_cm.pdf}
    \caption{SI2 opa1 lr04 bs32 en15 dsmel04 wLTBall noval cm}%
    \label{fig:SI2_opa1_lr04_bs32_en15_dsmel04_wLTBall_noval_cm}
\end{figure*}

\begin{figure*}[t!]
    \centering
    \includegraphics[width=0.9\linewidth]{ATT_ct02_dr01_ks01_lu01_qt01_dw01_opa1_lr03_bs02_en02_dsmel04_wLTBall_cm.pdf}
    \caption{ATT ct02 dr01 ks01 lu01 qt01 dw01 opa1 lr03 bs02 en02 dsmel04 wLTBall cm}%
    \label{fig:ATT_ct02_dr01_ks01_lu01_qt01_dw01_opa1_lr03_bs02_en02_dsmel04_wLTBall_cm}
\end{figure*}

% https://stackoverflow.com/a/15221702/2237151
% find . -type d -print0 | while read -d '' -r dir; do files=("$dir"/*); printf "%s & %d\n" "$dir" "${#files[@]}"; done

\begin{table}[h!]
    \centering
    \caption{Task composition.
    Each task is expanded by adding the
    \texttt{\_other\_ltts}
    category,
    and is referred as
    \texttt{LT\_\{task\_key\}}.
    The second column shows the number of samples for that word.
    }
    \label{tab:task_word_composition}
    \begin{tabular}{|c|c|ccccccc|}
        \hline
        Words & $\#$          &f1 &f2 &dir&num&k1 &w2 &all \\
        \hline
        backward & 1664       &   & x & x &   &   &   & x  \\
        bed & 2014            &   &   &   &   &   &   & x  \\
        bird & 2064           &   &   &   &   &   &   & x  \\
        cat & 2031            &   &   &   &   &   &   & x  \\
        \hline
        dog & 2128            &   &   &   &   &   &   & x  \\
        down & 3917           &   &   & x &   & x & x & x  \\
        eight & 3787          &   & x &   & x &   & x & x  \\
        five & 4052           &   &   &   & x &   & x & x  \\
        \hline
        follow & 1579         &   &   &   &   &   &   & x  \\
        forward & 1557        &   &   & x &   &   &   & x  \\
        four & 3728           &   &   &   & x &   & x & x  \\
        go & 3880             &   & x &   &   & x & x & x  \\
        \hline
        happy & 2054          & x &   &   &   &   &   & x  \\
        house & 2113          &   &   &   &   &   &   & x  \\
        learn & 1575          & x &   &   &   &   &   & x  \\
        left & 3801           &   &   & x &   & x & x & x  \\
        \hline
        marvin & 2100         &   &   &   &   &   &   & x  \\
        nine & 3934           &   &   &   & x &   & x & x  \\
        no & 3941             &   &   &   &   & x & x & x  \\
        off & 3745            &   &   &   &   & x & x & x  \\
        \hline
        on & 3845             &   &   &   &   & x & x & x  \\
        one & 3890            &   &   &   & x &   & x & x  \\
        right & 3778          &   &   & x &   & x & x & x  \\
        seven & 3998          &   &   &   & x &   & x & x  \\
        \hline
        sheila & 2022         &   &   &   &   &   &   & x  \\
        six & 3860            &   &   &   & x &   & x & x  \\
        stop & 3872           &   &   &   &   & x & x & x  \\
        three & 3727          &   &   &   & x &   & x & x  \\
        \hline
        tree & 1759           &   &   &   &   &   &   & x  \\
        two & 3880            &   &   &   & x &   & x & x  \\
        up & 3723             &   &   & x &   & x & x & x  \\
        visual & 1592         & x &   &   &   &   &   & x  \\
        \hline
        wow & 2123            & x &   &   &   &   &   & x  \\
        yes & 4044            &   & x &   &   & x & x & x  \\
        zero & 4052           &   &   &   & x &   & x & x  \\
        \_other\_ltts & 4954  &   &   &   &   &   &   &    \\
        \hline
    \end{tabular}
\end{table}

\begin{table}[h!]
% \begin{table}[H]
    \centering
    \caption{
    Values used to generate the mel spectrograms. The dataset ``mela1'' also
has the parameter \texttt{fmin}$=40$.}
    \label{tab:mel_values}
    \begin{tabular}{|c|cccc|}
        \hline
        % name & \texttt{n\_mel} & \texttt{n\_fft} & \texttt{hop\_length} & shape \\
        name & n\_mel & n\_fft & hop\_length & shape \\
        \hline
        mel01 & 128 & 2048 & 512   & (128, 32) \\
        mel02 & 64  & 4096 & 1024  & (64, 16) \\
        mel03 & 64  & 2048 & 512   & (64, 32) \\
        mel04 & 64  & 1024 & 256   & (64, 64) \\
        mel05 & 128 & 1024 & 128   & (128, 128) \\
        mel06 & 128 & 1024 & 256   & (128, 64) \\
        mel07 & 128 & 2048 & 256   & (128, 64) \\
        mel08 & 128 & 512  & 256   & (128, 64) \\
        mel09 & 128 & 512  & 128   & (128, 128) \\
        mel10 & 128 & 2048 & 128   & (128, 128) \\
        mel11 & 128 & 256  & 128   & (128, 128) \\
        mel12 & 128 & 4096 & 256   & (128, 64) \\
        mel13 & 128 & 512  & 256   & (128, 64) \\
        mel14 & 128 & 256  & 256   & (128, 64) \\
        mel15 & 128 & 3072 & 256   & (128, 64) \\
        mela1 & 80  & 1024 & 128   & (80, 128) \\
        \hline
    \end{tabular}
\end{table}

% \begin{table}[H]
\begin{table}[h!]
    \centering
    \caption{Values used to generate the MFCC spectrograms}
    \label{tab:mfcc_values}
    \begin{tabular}{|c|cccc|}
        \hline
        % name & \texttt{n\_mfcc} & \texttt{n\_fft} & \texttt{hop\_length} & shape \\
        name & n\_mfcc & n\_fft & hop\_length & shape \\
        \hline
        mfcc01 & 20  & 2048 & 512  & (20, 32) \\
        mfcc02 & 40  & 2048 & 512  & (40, 32) \\
        mfcc03 & 40  & 2048 & 256  & (40, 64) \\
        mfcc04 & 80  & 1024 & 128  & (80, 128) \\
        mfcc05 & 10  & 4096 & 1024 & (10, 16) \\
        mfcc06 & 128 & 1024 & 128  & (128, 128) \\
        mfcc07 & 128 & 512  & 128  & (128, 128) \\
        mfcc08 & 128 & 2048 & 128  & (128, 128) \\
        \hline
    \end{tabular}
\end{table}

% \begin{table}[H]
\begin{table}[h!]
    \centering
    \caption{Values used to compose the spectrogams}
    \label{tab:compose_values}
    \begin{tabular}{|c|cc|}
        \hline
        name & left spectrogam & right spectrogam \\
        \hline
        melc1 & mel06 & mel08 \\
        melc2 & mel07 & mel12 \\
        melc3 & mel13 & mel14 \\
        melc4 & mel13 & mel15 \\
        \hline
    \end{tabular}
\end{table}

% \begin{table}[H]
\begin{table}[h!]
    \centering
    \caption{Dataset stacked to obtain a 3-channel image}
    \label{tab:ch3_values}
    \begin{tabular}{|c|ccc|}
        \hline
        tag & spec1 & spec2 & spec3 \\
        \hline
        01 & mel05  & mel09  & mel10 \\
        02 & mel05  & mel10  & mfcc07 \\
        03 & mfcc06 & mfcc07 & mfcc08 \\
        04 & mel05  & mfcc06 & melc1 \\
        05 & melc1  & melc2  & melc4 \\
        \hline
    \end{tabular}
\end{table}

\begin{table*}[h!]
    \centering
    \caption{Values used to augment the dataset. The first lines list the parameters used to compute the spectrograms.}
    \label{tab:aug_values}
    \begin{tabular}{|c|cccccc|}
        \hline
        & mel\_kwargs & n\_mel & n\_fft & hop\_length & fmin & fmax \\
        \hline
        \hline
        & mel\_01 & 64 & 1024 & 256 & 40 & 8000  \\
        & mel\_02 & 128 & 2046 & 512 & 40 & 8000  \\
        & mel\_03 & 64 & 1024 & 256 & default & default  \\
        & mel\_05 & 128 & 1024 & 128 & default & default  \\
        \hline
        \hline
        aug name & max\_time\_shifts & stretch\_rate & mel\_kwargs & num\_landmarks & max\_warp\_time & max\_warp\_freq\\
        \hline
        \hline
        aug01 &  [1600, 3200] & [0.8, 1.2] & mel\_01 & 3 & 5 & 6 \\
        \hline
        aug02 &  [] & [] & mel\_02 & 3 & 5 & 5 \\
        aug03 &  [] & [] & mel\_02 & 3 & 5 & 0 \\
        aug04 &  [] & [] & mel\_02 & 3 & 0 & 5 \\
        aug05 &  [] & [] & mel\_02 & 3 & 0 & 0 \\
        \hline
        aug06 &  [] & [] & mel\_01 & 3 & 5 & 5 \\
        aug07 &  [] & [] & mel\_01 & 3 & 5 & 0 \\
        aug08 &  [] & [] & mel\_01 & 3 & 0 & 5 \\
        aug09 &  [] & [] & mel\_01 & 3 & 0 & 0 \\
        \hline
        aug10 &  [] & [] & mel\_03 & 3 & 5 & 5 \\
        aug11 &  [] & [] & mel\_03 & 3 & 5 & 0 \\
        aug12 &  [] & [] & mel\_03 & 3 & 0 & 5 \\
        aug13 &  [] & [] & mel\_03 & 3 & 0 & 0 \\
        \hline
        aug14 &  [] & [] & mel\_03 & 4 & 2 & 2 \\
        aug15 &  [] & [] & mel\_03 & 4 & 2 & 0 \\
        aug16 &  [] & [] & mel\_03 & 4 & 0 & 2 \\
        aug17 &  [] & [] & mel\_03 & 4 & 0 & 0 \\
        \hline
        aug18 &  [] & [] & mel\_05 & 3 & 5 & 5 \\
        aug19 &  [] & [] & mel\_05 & 3 & 5 & 0 \\
        aug20 &  [] & [] & mel\_05 & 3 & 0 & 5 \\
        aug21 &  [] & [] & mel\_05 & 3 & 0 & 0 \\
        \hline
    \end{tabular}
\end{table*}


\end{document}
